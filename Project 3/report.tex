\documentclass[a4paper,oneside,12pt]{article}
 
\usepackage[english]{babel}
\usepackage{newtxtext}
\usepackage{substitutefont}

\usepackage{amsbsy, amsmath, amsfonts}
\usepackage{siunitx}
\usepackage{graphicx}

\newcommand{\spr}[1]{\ensuremath{^\textrm{#1}}}
\newcommand{\sub}[1]{\ensuremath{_\textrm{#1}}}

\usepackage{multirow}
\usepackage{booktabs}
\usepackage{float}

\usepackage[]{hyperref}
\usepackage[shortlabels]{enumitem} 

\usepackage{natbib}
\bibliographystyle{alpha}
\setcitestyle{super,open={[},close={]}}

\newcommand{\skipline}{\hfill \break \\}
\newcommand{\rom}[1]{\uppercase\expandafter{\romannumeral #1\relax}}

\usepackage{caption}
\captionsetup[figure]{labelfont=it,textfont=it}
\captionsetup[table]{labelfont=it,textfont=it}
\usepackage{subfig}

\usepackage[strict]{changepage}

\usepackage[left=2.5cm,right=2.5cm,top=2.5cm,bottom=2.5cm,includehead,includefoot,headheight=16pt]{geometry}

\usepackage{fancyhdr}
\fancyhf{}
\renewcommand{\headrulewidth}{0pt} 
\fancyhead[L]{\nouppercase{\leftmark}}
\cfoot{\thepage}
\pagestyle{fancy}


\usepackage{setspace}
\setstretch{1.5}

\usepackage{emptypage}
\usepackage{amsmath}
\usepackage{xfrac}
\usepackage{amssymb}
\usepackage{cancel}
\usepackage{graphicx}
\usepackage{mathdots}
\usepackage[dvipsnames]{xcolor}

\author{Yasmin Bouhlada, Annalisa Dettori, Andrea Schinoppi}
\title{Report third project}
\date{$17^th$ January 2024}

\setcounter{section}{1}
\begin{document}

\makeatletter
    \begin{titlepage}
        \begin{center}
            \includegraphics[width=0.7\linewidth]{Bielefeld_University.png}\\[4ex]
            {\huge \bfseries  \@title }\\[2ex] 
            {\large  \@author}\\[50ex] 
            {\large \@date}
        \end{center}
    \end{titlepage}
\makeatother
\thispagestyle{empty}
\newpage

\section{Introduction}

\section{Data Set}

\subsection{First data set}

The first data set was made of ten columns $X$ for the independent variable and one column for the dependent variable $y$. The data set had $1000$ numeric samples and it has no NaNs. We didn't know what the variables represented, but we could resume the main information of the variables with this table:  

\begin{table}[H]
\centering
\begin{tabular}{|c|c|c|c|c|}
\hline
\empty & mean & std dev & min & max\\
\hline
X & $0.01$ & $4.89$ & $-9.99$ & $9.98$\\
\hline
y & $0.5$ & $0.5$ & $0$ & $1$\\
\hline

\end{tabular}
\end{table}


\subsection{Second data set}

The first data set was made of thirteen columns $X$ for the independent variable and one column for the dependent variable $y$. The data set had $110$ numeric samples and it has no NaNs. We didn't know what the variables represented, but we could resume the main information of the variables with this table:  

\begin{table}[H]
\centering
\begin{tabular}{|c|c|c|c|c|}
\hline
\empty & mean & std dev & min & max\\
\hline
X & $0.01$ & $4.89$ & $-9.99$ & $9.98$\\
\hline
y & $0.5$ & $0.5$ & $0$ & $1$\\
\hline

\end{tabular}
\end{table}

FARETABELLAAA

\subsection{Third data set}

The third data set (named \textit{real world data}) was a real data set regarding the prices of some houses. It was made of ten columns $X_1,\;.\;.\;.\;,X_{10}$ for the independent variables and one column for the dependent variable $y$. The data set had $1095$ numeric samples and it had no NaNs.

The labels of the variables were: "LotArea", "TotalBsmtSF", "1stFlrSF", "2ndFlrSF", "GrLivArea", "WoodDeckSF", "OpenPorchSF", "3SsnPoarch", "ScreenPorch" and "PoolArea".

We could resume the main information of the variables with this table:  

\begin{table}[H]
\centering
\begin{tabular}{|c|c|c|c|c|c|c|c|c|c|c|}
\hline
\empty & $X_1$ & $X_2$ & $X_3$ & $X_4$ & $X_5$ & $X_6$ & $X_7$ & $X_8$ & $X_9$ & $X_{10}$\\
\hline
mean & $10722.41$ & $1159.84$ & $0$ & $338.71$ & $1505.13$ & $91,06$ & $47.26$ & $2.78$ & $15.09$  & $2.14$\\
\hline
std dev & $11054.40$ & $376.46$ & $34900$ & $432.04$ & $514.24$ & $120.64$ & $66.79$ & $25.18$ & $56.55$ & $35.79$\\
\hline
min & $1300$ & $0$ & $343$ & $0$ & $334$ & $0$ & $0$ & $0$ & $0$ & $0$\\
\hline
max & $215245$ & $3206$ & $3228$ & $1872$ & $4676$ & $670$ & $547$ & $407$ & $480$ & $738$\\
\hline

\end{tabular}
\end{table}

\section{Foundations}

\subsection{Methods}

\subsubsection{Filter methods, wrapper methods and unbedded methods}

For variables selection what we decided to do was to apply three methods: filter methods, wrapper methods and unbedded methods. 

As filter method, we used the \textit{F measure}. We selected both $2$ and $6$ features:

\subsection{Evaluation}

\subsubsection{Filter method}

With the \textit{F measure} we selected the following features 

\begin{itemize}
\item Two features: \texttt{['TotalBsmtSF' 'GrLivArea']} 
\item Six features:  \texttt{['2ndFlrSF' 'OpenPorchSF' 'WoodDeckSF' '1stFlrSF' 'TotalBsmtSF'
 'GrLivArea']}
\end{itemize}

that provided the respective results for the $R^2$:
\begin{table}[H]
\centering
\begin{tabular}{|c|c|}
\hline
\empty & $R^2$ \\
\hline
2 features & $0.22$ \\
\hline
6 features & $0.47$ \\
\hline

\end{tabular}
\end{table}

\end{document}